%%% -*-LaTeX-*-
%%% lumpedchains.tex.orig
%%% Prettyprinted by texpretty lex version 0.02 [21-May-2001]
%%% on Thu Nov  4 08:43:49 2021
%%% for Steven R. Dunbar (sdunbar@family-desktop)

\documentclass[12pt]{article}

\input{../../../../etc/macros} %\input{../../../../etc/mzlatex_macros}
\input{../../../../etc/pdf_macros}

\bibliographystyle{plain}

\begin{document}

\myheader \mytitle

\hr

\sectiontitle{Lumped Markov Chains}

\hr

\usefirefox

\hr

% \visual{Study Tip}{../../../../CommonInformation/Lessons/studytip.png}
% \section*{Study Tip}

% \hr

\visual{Rating}{../../../../CommonInformation/Lessons/rating.png}
\section*{Rating} %one of
% Everyone: contains no mathematics.
% Student: contains scenes of mild algebra or calculus that may require guidance.
Mathematically Mature:  may contain mathematics beyond calculus with
proofs. % Mathematicians Only: prolonged scenes of intense rigor.

\hr

\visual{Section Starter Question}{../../../../CommonInformation/Lessons/question_mark.png}
\section*{Section Starter Question}

\hr

\visual{Key Concepts}{../../../../CommonInformation/Lessons/keyconcepts.png}
\section*{Key Concepts}

\begin{enumerate}
    \item
        Denote the transition probability of the Markov chain \( X_n \)
        from state \( x_i \) to state \( x_j \), for \( i, j = 1, \dots,
        r \), by \( p_{ij} \).  A necessary and sufficient condition for
        the Markov chain \( X \) to be lumpable with respect to the
        partition \( S \) is that for every pair of sets \( E_{\xi} \)
        and \( E_{\eta} \), \( sum_{x_k \in E_{\eta}} p_{ik} \) has the
        same value for every \( x_i \in E_{\xi} \).  These common values
        form the transition probabilities \( p_{\xi, \eta} \) for the
        lumped chain.
    \item
        if \( P \) is the transition probability matrix of the Markov
        chain \( X_n \), then \( X_n \) is lumpable with respect to the
        partition \( \mathcal{S} \), if and only if \( VUPV = PV \).
    \item
        Let \( \delta = \| VUP - P \|_{\pi} \).  Then
        \[
            \| (VUP)^n - P^n \|_{\pi} < K(n) < \hat{K} \delta
        \] with \( K(n) = n \abs{\lambda_2}^{n-1} \), where \( \lambda_2
        \) is the second largest eigenvalue of \( P \) and \( \hat{K} =
        -1/(\lambda_2 \EulerE \log(\lambda_2)) \).
\end{enumerate}

\hr

\visual{Vocabulary}{../../../../CommonInformation/Lessons/vocabulary.png}
\section*{Vocabulary}
\begin{enumerate}
    \item
        Let \( X_n \) be a Markov chain with state space \( \mathcal{X}
        = \set{x_1 x_2, \dots x k} \) and initial distribution \( \xi \).
        Given a partition \( \bar{S} = \set{E_1, E_2, \dots, E_v} \) of
        the state space \( S \), a new chain \( \bar{X}_n \) can be
        defined as follows:  At the \( j \)th step, the state of the new
        chain is the set \( E_k \) when \( E_k \) contains the state of
        the \( j \)th step of the original chain.  Assign the transition
        probabilities for \( \bar{X}_n \) as follows: The initial
        distribution is
        \[
            \Prob{\bar{X}_0 = E_i} = \Probsub{X_0 \in E_i}.
        \] Given the initial state, the transition probability for step \(
        1 \) is
        \[
            \Prob{\bar{X}_1 = E_j \given \bar{X}_0 = E_i} = \Probsub{\xi}
            {X_0 \in E_i }.
        \] In general for the \( n \)th step
        \[
            \Prob{\bar{X}_1 = E_t \given \bar{X}_{n-1} = E_{s_{n-1}},
            \bar{X}_{n-2} = E_{s_{n-2}}, \dots, \bar{X}_1 = E_{s_1},
            \bar{X}_0 = E_i} = \Probsub{\xi}{X_0 \in E_i \given X_{n-1}
            \in E_{s_{n-1}}, X_{n-2} \in E_{s_{n-2}}, \dots, X_1 \in E_{s_1},
            X_0 \in E_i}.
        \] Call this new chain \( \bar{X}_n \), a \defn{lumped chain}%
        \index{lumped chain}
        of the Markov chain \( X_n \).
    \item
        A Markov chain with \( X_n \) state space with state space \(
        \mathcal{X} \) is said to be \defn{lumpable}%
        \index{lumpable}
        with respect to a partition \( S \) of \( \mathcal{X} \) if for
        every starting distribution \( X_0 \) the lumped chain \( \bar{X}_n
        \) is a Markov chain with state space \( S \) and the associated
        transition probabilities do not depend on the choice of \( X_0 \).
    \item
        The \defn{distributing matrix}%
        \index{distributing
  matrix}
        \( U \) is the \( v \times k \) matrix with entries
        \[
            U_{ij} =
            \begin{cases}
                \pi_j/\sum_{k \in E_j} \pi_k & x_k \in E_j \\
                0 & \text{otherwise}.
            \end{cases}
        \] The rows of the distributing matrix are the stationary
        distribution restricted to \( E_j \) and renormalized so its
        entries add to \( 1 \).
    \item
        Let the \defn{collecting matrix}%
        \index{collecting matrix}
        \( V \) be the \( k \times v \) matrix with the \( j \)th
        column, \( j = 1, 2, \dots, v \), is a vector with \( 1 \)s in
        the components corresponding to states in \( E_j \), and \( 0 \)
        elsewhere.  The collecting matrix specifies the lumped
        probability distribution \( \hat{p} = p V \) on the partition \(
        E \).
\end{enumerate}

\hr

\visual{Mathematical Ideas}{../../../../CommonInformation/Lessons/mathematicalideas.png}
\section*{Mathematical Ideas}

It is often convenient to partition the states of a chain into
aggregates or lumps and to view the dynamics at a coarser level as the
system of interest moves among the lumps.  Examples of such aggregation
include a weather model lumping the states ``rain'' and ``snow'' into
one state called ``bad weatehr'' or a physical model aggregating the
microstates of a physical system into so-called coarse grained
``mesostates'' each representing many microstates.  Although the
dynamics moving among the lumps is not necessarily Markovian in general,
there is a natural choice for a Markov chain model on the set of lumped
states.  This choice matches the time evolution of the original unlumped
chain started at equilibrium. This provides bounds the error of the
dynamics predicted by the lumped chain considered as a model of the
unlumped chain.  The goal in this section is to analyze the accuracy of
such coarse grained models compared to the exact microscopic behavior,
i.e.\ to bound the error in a coarse grained description.

\subsection*{Lumpable Markov Chains}

%% https://www.math.pku.edu.cn/teachers/yaoy/Fall2011/Kemeny-Snell_Chapter6.3-4.pdf
%% https://web.nmsu.edu/~jtian/PB/2006-3.pdf
%% https://www.ncbi.nlm.nih.gov/pmc/articles/PMC3207820/
%% http://www.sci.sdsu.edu/~salamon/OLPpublished.pdf

\begin{definition}
    Let \( X_n \) be a Markov chain with state space \( \mathcal{X} =
    \set{x_1 x_2, \dots x_k} \) and initial distribution \( \xi \).
    Given a partition \( \bar{S} = \set{E_1, E_2, \dots, E_v} \) of the
    state space \( S \), a new chain \( \bar{X}_n \) can be defined as
    follows:  At the \( j \)th step, the state of the new chain is the
    set \( E_k \) when \( E_k \) contains the state of the \( j \)th
    step of the original chain.  Assign the transition probabilities for
    \( \bar{X}_n \) as follows: The initial distribution is
    \[
        \Prob{\bar{X}_0 = E_i} = \Probsub{\xi}{X_0 \in E_i}.
    \] Given the initial state, the transition probability for step \( 1
    \) is
    \[
        \Prob{\bar{X}_1 = E_j \given \bar{X}_0 = E_i} = \Probsub{\xi}{X_0
        \in E_i }.
    \] In general for the \( n \)th step
    \[
        \Prob{\bar{X}_n = E_t \given \bar{X}_{n-1} = E_{t_{n-1}}, \bar{X}_
        {n-2} = E_{t_{n-2}}, \dots, \bar{X}_1 = E_{t_1}, \bar{X}_0 = E_i}
        = \Probsub{\xi}{X_0 \in E_i \given X_{n-1} \in E_{t_{n-1}}, X_{n-2}
        \in E_{t_{n-2}}, \dots, X_1 \in E_{t_1}, X_0 \in E_i}.
    \] Call this new chain \( \bar{X}_n \), a \defn{lumped chain}%
    \index{lumped chain}
    of the Markov chain \( X_n \).
\end{definition}

A lumped chain of a given Markov chain need not be a Markov chain in
general.  [Need example]

\begin{definition}
    A Markov chain with \( X_n \) state space with state space \(
    \mathcal{X} \) is said to be \defn{lumpable}%
    \index{lumpable}
    with respect to a partition \( S \) of \( \mathcal{X} \) if for
    every starting distribution \( \xi \) the lumped chain \( \bar{X}_n \)
    is a Markov chain with state space \( S \) and the associated
    transition probabilities do not depend on the choice of \( \xi \).
\end{definition}

The following theorem gives a necessary and sufficient for a Markov
chain to be lumpable.

\begin{theorem}
    Denote the transition probability of the Markov chain \( X_n \) from
    state \( x_i \) to state \( x_j \), for \( i, j = 1, \dots, r \), by
    \( p_{ij} \).  A necessary and sufficient condition for the Markov
    chain \( X \) to be lumpable with respect to the partition \( S \)
    is that for every pair of sets \( E_i \) and \( E_j \), \( sum_{x_k
    \in E_j} p_{ik} \) has the same value for every \( x_i \in E_j \).
    These common values form the transition probabilities \( \hat{p}_{ij}
    \) for the lumped chain.
\end{theorem}

\begin{proof}
    \begin{enumerate}
        \item
            Necessity:
            \begin{enumerate}
                \item
                    For the chain to be lumpable, it is necessary that
                    \[
                        \Prob{\bar{X}_1 = E_j \given \bar{X}_0 = E_i} =
                        \Probsub{\xi}{X_0 \in E_i }
                    \] be the same for every \( \xi \) for which it is
                    defined.  Call this common value \( \hat{p}_{ij} \).
                \item
                    In particular this must be the same for \( \xi \)
                    having a \( 1 \) in its \( k \)th component for
                    state \( x_k \in E_i \). Hence \( p_{kE_j} = \Prob{x_i
                    \in A_j} = \hat{p}_{ij} \).
                \item
                    Thus the condition given is necessary.
            \end{enumerate}
        \item
            Sufficiency:
            \begin{enumerate}
                \item
                    The proof must show that if the condition is
                    satisfied, the probability
                    \[
                        \Prob{\bar{X}_n = E_t \given \bar{X}_{n-1} = E_{t_
                        {n-1}}, \bar{X}_{n-2} = E_{t_{n-2}}, \dots, \bar
                        {X}_1 = E_{t_1}, \bar{X}_0 = E_i} = \Probsub{\xi}
                        {X_0 \in E_i \given X_{n-1} \in E_{t_{n-1}}, X_{n-2}
                        \in E_{t_{n-2}}, \dots, X_1 \in E_{t_1}, X_0 \in
                        E_i}
                    \] depends only on \( E_t \) and \( E_{t_{n-1}} \).
                \item
                    This conditional probability may be written in the
                    form \( \Probsub{\xi'}{x_1 \in E_t} \) where \( \xi'
                    \) is a vector with non-zero components only in the
                    sates of \( E_{t_{n-1}} \).
                \item
                    This probability depends on \( \xi \) and on the
                    first \( n \) outcomes.
                \item
                    However, if \( \Probsub{k}{x_1 \in E_t} = \hat{p}_{t,
                    t_{n_1}} \) for all \( x_k \in E_t \), then it is
                    clear that \( \Probsub{\xi'}{x_1 \in E_t} = \hat{p}_
                    {st} \).
                \item
                    Thus the probability depends only on \( E_s \) and \(
                    E_t \).
            \end{enumerate}
    \end{enumerate}
\end{proof}

\begin{example}
    Recall the example of the weather in the land of Oz, where it is
    rainy, nice or snowy, with probability transition matrix
    \[
        P = \bordermatrix{ & R & N & S \cr
        R & 1/2 & 1/4 & 1/4 \cr
        N & 1/2 & 0 & 1/2 \cr
        S & 1/4 & 1/4 & 1/2 }.
    \] Now consider lumping the weather states into only ``good'' days
    with nice weather and ``bad'' days with rainy or snowy weather.
    That is, choose the partition \( S = \set{ \set{N}, \set{R,S}} =
    \set{G, B} \). The probabilities of moving from \( R \) to \( N \)
    and from \( S \) to \( N \) are the same so the condition for
    lumpability is satisfied.  The new transition matrix is
    \[
        P = \bordermatrix{ G & B \cr
        G & 0 & 1 \cr
        B & 1/4 & 3/4}.
    \]

    On the other hand, notice that the condition for lumpability is not
    satisfied for the partition \( S = \set{ \set{R}, \set{N, S}} \)
    since \( p_{N S_1} = p_{NR} = 1/2 \) and \( p_{N S_1} = p_{SR} = 1/4
    \).
\end{example}

A matrix formulation of this theorem is useful. Introduce some
notations. Associated with a partition \( \mathbf{S} = \set{E_1, E_2,
\dots, E_v} \) of the finite state space \( \mathcal{X} = \set{x_1, x_2,
\dots, x_k} \), of the underlying Markov chain \( X_n \), introduce two
useful matrices.  The \defn{distributing matrix}%
\index{distributing
  matrix}
\( U \) is the \( v \times k \) matrix with entries
\[
    U_{ij} =
    \begin{cases}
        \pi_j/\sum_{k \in E_j} \pi_k & x_k \in E_j \\
        0 & \text{otherwise}.
    \end{cases}
\] The rows of the distributing matrix are the stationary distribution
restricted to \( E_j \) and renormalized so its entries add to \( 1 \).
For any probability distribution \( q \) in the lumped space, the image \(
qU \) is locally equilibrated, i.e.\ \( qU \) restricted to any element \(
E_j \) of the partition equals the stationary distribution restricted to
\( E_j \) and renormalized to add to \( q_j \).  In particular, with \(
\hat{\pi} = \pi V \)
\[
    U_{ij} = V_{ji} \frac{\pi_j}{\hat{\pi}_{i}}, i = 1, \dots, k, j = 1,
    \dots, v.
\] whose \( i \)th row, \( \i = 1, 2, \dots, v \), is the probability
vector having equal values for states in \( E_{\xi} \) and \( 0 \)
elsewhere.

Note that the rows of \( PV \) corresponding to the elements in the same
set of the partition are the same.  This will be true in general for a
chain satisfying the condition for lumpability.  The matrix \( U \)
simply removes this duplication of rows.  The choice of \( U \) is not
unique, all that is needed is that the \( i \)th row should be a
probability vector with non-zero components only for states in \( E_i \).
Sometimes for simplicity and convenience, choose the vector with equal
components.

Let the \defn{collecting matrix}%
\index{collecting matrix}
\( V \) be the \( k \times v \) matrix with the \( j \)th column, \( j =
1, 2, \dots, v \), is a vector with \( 1 \)s in the components
corresponding to states in \( E_j \), and \( 0 \) elsewhere.  The
collecting matrix specifies the lumped probability distribution \( \hat{p}
= p V \) on the partition \( E \).  Then \( UV = I \) since \( V \)
simply collects back exactly what \( U \) distributes.  The lumped
transition matrix is \( \hat{P} = UPV \).

Next consider the local equilibration operator \( VU \).  The name is
justified by considering its action.  Starting from any distribution \(
p \) on \( \mathcal{X} \), \( VU \) collects the probability in each \(
E_j \), then redistributes this much probability \( pV \) among the
lumped states in the partition member \( E_j \) as \( (pC)_j \hat{\pi}
\hat{\pi}_i^j \).  In particular,
\[
    \pi = \pi VU.
\]

\begin{example}
    In the previous lumped land of Oz weather example
    \[
        \hat{P} =
        \begin{pmatrix}
            0 & 1 & 0 \\
            1/2 & 0 & 1/2
        \end{pmatrix}
        \begin{pmatrix}
            1/2 & 1/4 & 1/4 \\
            1/2 & 0 & 1/2 \\
            1/4 & 1/4 & 1/2
        \end{pmatrix}
        \begin{pmatrix}
            0 & 1 \\
            1 & 0 \\
            0 & 1
        \end{pmatrix}
        =
        \begin{pmatrix}
            0 & 1 \\
            1/4 & 3/4
        \end{pmatrix}
        .
    \]
\end{example}

It is convenient to assume that the state are ordered so that those in \(
E_1 \) come first, those in \( E_{2} \) come next and so on with those
in \( E_r \) coming last.  From here on, assume that this ordering of
states has been made.

\begin{theorem}
    if \( P \) is the transition probability matrix of the Markov chain \(
    X_n \), then \( X_n \) is lumpable with respect to the partition \(
    \mathcal{S} \), if and only if \( VUPV = PV \).
\end{theorem}

\begin{proof}
    \begin{enumerate}
        \item
            \( \Rightarrow \)
            \begin{enumerate}
                \item
                    The matrix \( VU \) has the block matrix form
                    \[
                        VU =
                        \begin{pmatrix}
                            W_1 & 0 & 0 \\
                            0 & W_2 & 0 \\
                            0 & 0 & W_3
                        \end{pmatrix}
                    \] where \( W_1 \), \( W_2 \), and \( W_3 \) are
                    probability matrices.
                \item
                    The hypothesis that \( VUPV = PV \) means that the
                    columns of \( PV \) are fixed vectors of \( VU \).
                \item
                    Since the chain is lumpable, the probability of
                    moving from a state of \( E_i \) to the set \( E_j \)
                    is the same for all states of \( A_i \), hence the
                    components of a column of \( PV \) corresponding to \(
                    A_j \) are all the same.
                \item
                    Therefore the components form a fixed vector for \(
                    W_j \), this proves \( VUPV = PV \).
            \end{enumerate}
        \item
            \( \Leftarrow \)
            \begin{enumerate}
                \item
                    Assume \( VUPV = PV \).  Then the columns of \( PV \)
                    are fixed vectors for \( VU \).  But each \( W_j \)
                    is the transition change of an ergodic chain, so its
                    only fixed column vectors are of the form \( c
                    \mathbf{1} \).  Hence all the components of a column
                    of \( PV \) must be the same, so the chain is
                    lumpable.
            \end{enumerate}
    \end{enumerate}
\end{proof}

\begin{corollary}
    \[
        \hat{P}^n = U P^n V.
    \]
\end{corollary}

\begin{proof}
    From the theorem,
    \[
        \hat{P}^2 = U P V U P V = U P^2 V
    \] and the corollary follows by induction.
\end{proof}

\subsection*{Lumping Absorbing Chains}

Assume \( P \) is an absorbing chain.  Restrict attention to the case of
lumping only states of the same kind. That is, any subset of our
partition will contain only absorbing states or only non-absorbing
states.  Recall that the standard form for an absorbing chain is
\[
    P =
    \begin{pmatrix}
        I & 0 \\
        R & Q
    \end{pmatrix}
\] and write \( U \) in the form
\[
    U =
    \begin{pmatrix}
        U_1 & 0 \\
        0 & U_{2}
    \end{pmatrix}
\] where entries of \( U_1 \) refer to absorbing states and entries of \(
U_2 \) to non-absorbing states.  Similarly write \( V \) in the form
\[
    V
    \begin{pmatrix}
        V_1 & 0 \\
        0 & V_{2}
    \end{pmatrix}
    .
\] In terms of this decomposition, the matrix condition for lumpability
becomes
\begin{align*}
    V_1 U_1 V_1 &= V_1 \\
    V_2 U_2 R V_1 &= R V_1 \\
    V_2 U_2 Q V_2 &= Q V_2.
\end{align*}
Since \( U_1 V_1 = I \), the first condition is automatically satisfied.
The standard form for the transition matrix \( \hat{P} \) is
\[
    \hat{P} = UPV =
    \begin{pmatrix}
        U_1 & 0 \\
        0 & U_{2}
    \end{pmatrix}
    \begin{pmatrix}
        I & 0 \\
        R & Q
    \end{pmatrix}
    \begin{pmatrix}
        V_1 & 0 \\
        0 & V_{2}
    \end{pmatrix}
    . =
    \begin{pmatrix}
        V_1 & 0 \\
        0 & V_{2}
    \end{pmatrix}
    \begin{pmatrix}
        I & 0 \\
        U_2 R V_1 & U_2 Q V_2
    \end{pmatrix}
    .
\] Then
\begin{align*}
    \hat{R} &= U_2 R V_1 \\
    \hat{Q} &= U_2 Q V_2.
\end{align*}
From \( V_2 U_2 Q V_2 = Q V_2 \), obtain \( \hat{Q}^2 = U_2 Q V_2 U_2 Q
V_2 = U_2 Q^2 V_2 \).

From the infinite series representation for the fundamental matrix \( N \)
we have
\begin{align*}
    \hat{N} &= I + \hat{Q} + \hat{Q}^2 + \cdots \\
    &= U_2 I V_2+ U_2 Q V_2 + U_2 Q^2 V_2 + \cdots \\
    &= U_2 (I + Q + Q^2 + \cdots) \\
    &= U_2 N V_2.
\end{align*}
From this \( \hat{\tau} = U_2 N V_2 \mathbf{1} = U_2 N \mathbf{1} = U_2
\tau \)and
\begin{align*}
    \hat{B} = \hat{N} \hat{R} &= U_2 N V_2 U_2 R V_1 \\
    &= U_2 N R V_1 \\
    &= U_2 B V_1.
\end{align*}
So the important absorption quantities are all easily obtained for the
lumped chain from the original chain.

A consequence of \( \hat{\tau} = U_2 \tau \) is the following. Let \( E_j
\) be any non-absorbing set, and let \( x_i \) be a state in \( E_i \).
Choose the \( i \)th row of \( U_2 \) be a probability vector with \( 1 \)
in the \( x_k \) component.  This means \( t_i = t_\ell \) for all \( x_\ell
\) in \( E_j \).  When a chain is lumpable, the mean time to absorption
must be the same for all starting states in the same set \( E_j \)

\subsection*{Lumping Ergodic Chains}

Assume the ergodic chain \( X \) is lumpable for some partition \( S \).
The resulting chain will be also be ergodic.  Let \( \hat{A} \) be th
limiting matrix for the lumped chain.  Then
\begin{align*}
    \hat{A} &= \lim_{\nu \to \infty} \frac{\hat{P} + \hat{P}^2 + \cdots
    + \hat{P}^{\nu}}{\nu} \\
    &= \lim_{\nu \to \infty} \frac{UPV + UP^2V + \cdots + UP^{\nu}V}{\nu}
    \\
    & = UAV.
\end{align*}
This states that the components of \( \hat{\alpha} \) are obtained from \(
\alpha \) by simply adding the components of a given set.  From the
infinite series representation for the fundamental matrix \( \hat{Z} \)
\[
    \hat{Z} = U Z V.
\] In general, \( M \) and \( \hat{M} \) have no simple relationship.
However, the mean time to go from state \( x_i \) in \( E_j \) to \( E_{\ell}
\) in the original process is the same for all states in \( E_i \).
\subsection*{Examples}

\begin{example}
    Consider a random walk%
    \index{random walk}
    of a particle which moves along a straight line in unit steps.  Each
    step is \( 1 \) unit to the right with probability \( \frac{1}{2} \)
    and to the left with probability \( \frac{1}{2} \).  It moves until
    it reaches one of two extreme points which are absorbing
    boundaries.~%
    \index{absorption probability matrix}
    Assume that if the process reaches the boundary points, it remains
    there from that time on.  Figure~%
    \ref{fig:waitingtimeabsorbtion:randomwalkphasespace} has \( 9 \)
    states numbered from \( -4 \) to \( 4 \).  The absorbing boundary
    states are \( -4 \) and \( 4 \).

    In the small case the state space has \( 5 \) values with absorbing
    boundary states \( x_1 \) and \( x_5 \) and using the ordering for
    the standard form the transition matrix is
    \[
        P = \bordermatrix{ & s_1 & s_5 & s_2 & s_3 & s4 \cr
        s_1 1 & 0 & 0 & 0 & 0 \cr
        s_5 0 & 1 & 0 & 0 & 0 \cr
        s_2 1/2 & 0 & 0 1/2 & 0 \cr
        s_3 0 & 0 1/2 & 0 & 1/2 \cr
        s_4 & 0 & 0 1/2 & 0 \cr
        }.
    \] Take the partition \( S = \set{ \set{s_1, s_5}, \set{s_2, s_4},
    \set{s_3}} \).  For this partition the condition for lumpabiIity is
    satisfied, Notice that this would not have been the case if we have
    unequal probabilities for moving to left or right.

    It is easy to verify that
    \begin{align*}
        \hat{P} &=
        \begin{pmatrix}
            1 & 0 & 0 \\
            1/2 & 0 & 1/2 \\
            0 & 1 & 0
        \end{pmatrix}
        \\
        \hat{N} &=
        \begin{pmatrix}
            2 & 1 \\
            2 & 2
        \end{pmatrix}
        \\
        \hat{\tau} &= (3, 4)^T \\
        \hat{B} &= (1,1)^T.
    \end{align*}
\end{example}

\subsection*{Covariance}

It is possible to compute the covariance matrix of the lumped process.
The covariances are easily obtainable from \( C \) even if the original
process is not lumpable with respect to the partition, that is, the
lumped process is not a Markov chain.  In any case,
\[
    \hat{C} = \sum{s_k \in A_i, s_{\ell} \in A_j } c_{kl}.
\]

\begin{example}
    For the weather in the Land of Oz example.
    \begin{align*}
        \hat{A} &=
        \begin{pmatrix}
            0 & 1 & 0 \\
            1/2 & 0 & 1/2
        \end{pmatrix}
        \begin{pmatrix}
            2/5 & 1/5 & 2/5 \\
            2/5 & 1/5 & 2/5 \\
            2/5 & 1/5 & 2/5 \\
        \end{pmatrix}
        \begin{pmatrix}
            0 & 1 \\
            1 & 0 \\
            0 & 1
        \end{pmatrix}
        \\
        & =
        \begin{pmatrix}
            1/5 & 4/5 \\
            1/5 & 4/5
        \end{pmatrix}
        \\
        Z =
        \begin{pmatrix}
            0 & 1 & 0 \\
            1/2 & 0 & 1/2
        \end{pmatrix}
        \begin{pmatrix}
            86/75 & 13/75 & -14/75 \\
            6/75 & 63/75 & 6/75 \\
            -14/75 & 3/75 & 86/75 \\
        \end{pmatrix}
        \begin{pmatrix}
            0 & 1 \\
            1 & 0 \\
            0 & 1
        \end{pmatrix}
        \\
        & =
        \begin{pmatrix}
            63/75 & 12/75 \\
            3/75 & 72/75
        \end{pmatrix}
        \\
        \hat{C} =
        \begin{pmatrix}
            12/125 & -12/125 \\
            -12/125 & 12/125
        \end{pmatrix}
        \\
        M &= \bordermatrix{ & R & N & S \cr
        R & 5/2 & 4 & 10/3 \cr
        8/3 & 5 & 8/3 \cr
        10/3 & 4 & 5/2}.
    \end{align*}
    From the fundamental matrix \( \hat{Z} \) find that
    \[
        \hat{M} = \bordermatrix{ & N & B \cr
        N & 5 & 1 \cr
        B & 4 & 5/4 }.
    \] The mean time to reach \( N \) from either \( R \) or \( S \) is \(
    4 \).  Recall that \( N \) in the lumped process is a single element
    set.  This common value is the mean time in the lumped chain to go
    from \( B \) to \( N \).  Similarly, the value \( 5 \) is obtainable
    from \( M \).  The mean time to go from \( N \) to \( B \) is less
    than the mean time to go from \( N \) to either of the states \( B \)
    in the original process.
\end{example}

\subsection*{Bounding Lumping Errors}

Starting from a distribution \( x_0 \) on \( \mathcal{X} \), the time
evolution among the lumped states \( E_j \) is \( \hat{x}_{t} = x_0 P^t
V \), while the time evolution of the lumped Markov chain is \( y_t = x_0U
\hat{P}^t \).  The question is how different these two dynamics can be.

Define the norm \( \| \cdot \|_{\pi} \) by
\[
    \| v\|_{\pi} = \| v D_{\pi} \|_2 = \sqrt{\sum_{\nu} v_{\nu}^{2/\pi_\nu}}
\] where \( D_{\pi} =
\operatorname{diag}
(1/\sqrt{\pi_1}, \dots, 1/\sqrt{\pi_k}) \) and \( \| \cdot \|_2 \) is
the standard Euclidean \( 2 \)-norm.  The corresponding operator norm is
\begin{multline*}
    \| A \|_{\pi} = \max_{\|x\|_{\pi}=1}\|xA\|_{\pi} = \max_{\|xD_{\pi}\|_2
    = 1}\|x D_{\pi} D_{\pi}^{-1} A D_{\pi}\|_{2} = \\
    \max_{\|w\|_2 = 1}\|w D_{\pi}^{-1} A D_{\pi}\|_{2} = \| D_{\pi}^{-1}
    A D_{\pi}\|_{2}
\end{multline*}

The one step time difference is
\[
    \| x_0 V \hat{P} - x_0 G V \|_{\pi} = \| x_0 V U P V - x_0 P V \|_{\pi}
    = \| x_0 (VUP - P) V \|_{\pi}.
\] The \( n \) step time difference is
\begin{multline*}
    \| x_0 V \hat{P}^n - x_0 G^n V \|_{\pi} = \| x_0 V (U P V)^n - x_0 P^n
    V \|_{\pi} \\
    = \| x_0 VUPVUP \cdots PVUPPV - x_0 P^n V \|_{\pi} = \| x_0 (VUP)^n
    V - x_0 P^n V \|_{\pi} = \| x_0 ((VUP)^n - P^n) V \|_{\pi}.
\end{multline*}
The important operator for bounding the difference is \( VUP \). For
convenience, let \( VUP = H \).

The goal is to bound the \( n \)-step difference \( \| H^n - P^n \|_{\pi}
\) in terms of the \( 1 \)-step difference \( \| H - P\|_{\pi} \).  The
general behavior of the difference is that it can grow for a while, but
must eventually decline to \( 0 \) since by construction the lumped and
unlumped chain converge to the same equilibrium distribution. Bounding
how large the difference can get in terms of the \( 1 \)-step difference
is the next theorem.

\begin{theorem}
    Let \( \delta = \| VUP - P \|_{\pi} \).  Then
    \[
        \| (VUP)^n - P^n \|_{\pi} < K(n) < \hat{K} \delta
    \] with \( K(n) = n \abs{\lambda_2}^{n-1} \), where \( \lambda_2 \)
    is the second largest eigenvalue of \( P \) and \( \hat{K} = -1/(\lambda_2
    \EulerE \log(\lambda_2)) \).
\end{theorem}

\begin{proof}
    \begin{enumerate}
        \item
            First note that
            \[
                \| H^n - P^n \|_{\pi} = \|(H-P)H^{n-1}+P(H^{n-1} -P^{n-1}\|_
                {\pi}.
            \]
        \item
            Iterating,
            \begin{align*}
                \| H^n - P^n \|_{\pi} &= \| \sum_{\nu=0}^{n-1} P^k(H-P)H^
                {n-k-1}\|_{\pi} \\
                & \le \sum_{\nu=0}^{n-1} \| P^k(H-P)H^{n-k-1}\|_{\pi}.
            \end{align*}
        \item
            Recall that \( H \) and \( P \) have common stationary
            distribution \( \pi \).  Define the projection \( P_{\pi} =
            \mathbf{1}^T \pi \).  The complementary projection is \( P_{\sigma}
            = I - P_{\pi} \). \( P_{\sigma} \) projects any vector onto
            the subspace \( \Sigma = \setof{v}{v \mathbf{1}^T} \).  \(
            \Sigma \) is invariant under the action of any stochastic
            matrix.
        \item
            The decomposition into the subspaces gives the
            representation of \( P \) and \( H \):
            \begin{align*}
                P &= (P_{\pi} + P_{\sigma}) P = P_{\pi} + P_{\sigma}P \\
                H &= (P_{\pi} + P_{\sigma}) H = P_{\pi} + P_{\sigma}H.
                \\
            \end{align*}
        \item
            Then \( H - P = P_{\sigma}H - P_{\sigma}P \) and \( 0 = P_{\pi}
            P_{\sigma} P = P_{\sigma} P P_{\pi} = P_{\pi} P_{\sigma} H =
            P_{\sigma} H P_{\pi} \).
        \item
            Recall \( \delta = \| VUP - P \|_{\pi} \) and \( VUP = H \),
            so \( \| VUP - P \|_{\pi} = \| H - P \|_{\pi} = \|P_{\sigma}H
            - P_{\sigma}P\|_{\pi}\|_{\pi} = \delta \).
        \item
            For integers \( k \), \( n \)
            \begin{align*}
                & \| P^k(H-P)H^{n-k-1} \|_{\pi} \\
                & \qquad = \|(P_{\pi}+P_{\sigma}P)^k(P_{\sigma}H-P_{\sigma}P)
                (P_{\pi}+P_{\sigma}H)^{n-1-K}\|_{\pi} \\
                & \qquad = \|P_{\sigma}P)^k(P_{\sigma}H-P_{\sigma}P) (P_
                {\sigma}H)^{n-1-K}\|_{\pi} \\
                & \qquad = \| P_{\sigma} P \|_{\pi}^k \delta \|P_{\sigma}H\|_
                {\pi}^{n-k-1}.
            \end{align*}
        \item
            Use the special choice of norm, in particular, use the fact
            that for a matrix \( \| A \|_{\pi} = \| D_{\pi}^{-1} A D_{\pi}
            \|_{2} \).  By the assumption of reversibility, \( D_{\pi}^{-1}
            P D_{\pi} \) is a symmetric matrix.  Therefore, the \( 2 \)-norm
            is the dominant eigenvalue of the matrix.  In fact, as shown
            below the diagonal matrix \( U_{\pi} \) also symmetrizes \(
            P_{\pi} \), \( P_{\pi}P \), \( P_{\sigma}P \), and \( VU \).
            Thus each has norm equaling the dominant eigenvalue.
        \item
            Direct calculation shows \( \pi U_{\pi} = \mathbf{1} U_{\pi}^
            {-1} \).  Thus, \( D_{\pi}^{-1} P_{\pi} D_{\pi} = D_{\pi}^{-1}
            \mathbf{1}^{T} \pi D_{\pi} = (\mathbf{1} D_{\pi}^{-1})^T(\pi
            D_{\pi}) = (\pi D_{\pi})^T(\pi D_{\pi}) = ( (\pi D_{\pi})^T(\pi
            D_{\pi}))^T = (D_{\pi}^{-1} P_{\pi} D_{\pi})^T \).  This
            shows \( P_{\pi} \) is symmetrized by \( D_{\pi} \).
        \item
            So \( D_{\pi}^{-1} P D_{\pi} \) and \( D_{\pi}^{-1} P_{\pi}
            D_{\pi} \) are symmetric and commute with each other since \(
            P \) and \( P_{\pi} \) commute.  Thus, \( D_{\pi}^{-1} P_{\pi}
            P D_{\pi} = D_{\pi}^{-1} P_{\pi} D_{\pi} D_{\pi}^{-1} P D_{\pi}
            \) is symmetric.
        \item
            Then \( D_{\pi}^{-1} P_{\sigma} P D_{\pi} = D_{\pi}^{-1} (I-P_
            {\pi}) P D_{\pi} \) is symmetric as well.
        \item
            From \( U_{ij} = V_{ji} \frac{\pi_j}{\hat{\pi}_{i}} \), it
            follows that \( D = D_{\hat{\pi}}^2 V^T D_{\pi}^2 \) or
            equivalently \( D_{\hat{\pi}}^{-1} U D_{\pi} = D_{\hat{\pi}}
            V^T D_{\pi} \).
        \item
            Thus,
            \begin{align*}
                D_{\pi}^{-1} VU D_{\pi} &= (D_{\pi}^{-1} V D_{\hat{\pi}})
                (D_{\hat{\pi}}^{-1} U D_{\pi}) \\
                &= (D_{\pi}^{-1} V D_{\hat{\pi}})(D_{\hat{\pi}}^{-1} V^T
                D_{\pi}^{-1}) \\
                &= (D_{\pi}^{-1} V D_{\hat{\pi}})(D_{\pi}^{-1} V D_{\hat
                {\pi}}).
            \end{align*}
            As the product of a matrix and it transpose, \( VU \) is
            symmetrized by \( D_{\pi} \).
        \item
            Since \( D_{\pi}^{-1} P_{\sigma} P D_{\pi} \) is symmetric
            and based on the definition of the matrix norm \( \| P_{\sigma}
            P \|_{\pi} = \abs{\lambda_2} \) where \( \lambda_2 \) is the
            second-largest eigenvalue of \( P \).  To bound \( \| P_{\sigma}
            H \)
            \begin{multline*}
                \| P_{\sigma} H \|_{\pi} = \| VUP - P_{\sigma} \|_{\pi}
                = \| VUP - VUP_{\sigma}\|_{\pi} \\
                = \| VU(P - P_{\sigma})\|_{\pi} \le \|P_{\sigma} P\|_{\pi}
                \|VU\|_{\pi} = \abs{\lambda_2} \|VU\|_{\pi}.
            \end{multline*}
        \item
            Since \( VU \) is a stochastic matrix and\( D_{\pi} \)
            symmetrizes \( VU \), it follows that \( \| VU \|_{\pi} = 1 \).
        \item
            Thus the bound becomes \( \| H^n - P^n\|_{\pi} \le n \abs{\lambda_2}^
            {n-1} \delta = K(n) \delta \) where \( K(n) = n \abs{\lambda_2}^
            {n-1} \).
        \item
            Maximizing \( K(n) \) over \( n \) gives \( K(n) \le \hat{K}
            = \frac{-1}{\lambda_2 \EulerE \log(\lambda_2)} \).
    \end{enumerate}
\end{proof}

\visual{Section Starter Question}{../../../../CommonInformation/Lessons/question_mark.png}
\section*{Section Ending Answer}

\subsection*{Sources} This section is adapted from:

\nocite{}
\nocite{}

\hr

\visual{Algorithms, Scripts, Simulations}{../../../../CommonInformation/Lessons/computer.png}
\section*{Algorithms, Scripts, Simulations}

\subsection*{Algorithm}

\subsection*{Scripts}

%% \input{ _scripts}

\hr

\visual{Problems to Work}{../../../../CommonInformation/Lessons/solveproblems.png}
\section*{Problems to Work for Understanding}
\renewcommand{\theexerciseseries}{}
\renewcommand{\theexercise}{\arabic{exercise}}

% \begin{exercise}

% \end{exercise}
% \begin{solution}

% \end{solution}
% \begin{exercise}
%     \begin{enumerate}[label=(\alpha*)]
%     \item
% \end{enumerate}
% \end{exercise}
% \begin{solution}
%     \begin{enumerate}[label=(\alpha*)]
%     \item
% \end{enumerate}
% \end{solution}

\hr

\visual{Books}{../../../../CommonInformation/Lessons/books.png}
\section*{Reading Suggestion:}

\bibliography{../../../../CommonInformation/bibliography}

%   \begin{enumerate}
%     \item
%     \item
%     \item
%   \end{enumerate}

\hr

\visual{Links}{../../../../CommonInformation/Lessons/chainlink.png}
\section*{Outside Readings and Links:}
\begin{enumerate}
    \item
    \item
    \item
    \item
\end{enumerate}

\section*{\solutionsname} \loadSolutions

\hr

\mydisclaim \myfooter

Last modified:  \flastmod

\end{document}

%%% Local Variables:
%%% mode: latex
%%% TeX-master: t
%%% End:
